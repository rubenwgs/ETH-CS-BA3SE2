\documentclass[a4paper]{extarticle}
\usepackage[utf8]{inputenc}
\usepackage[a4paper, margin=1in]{geometry}

\usepackage{amssymb}
\usepackage{amsmath}
\usepackage{enumitem}
\usepackage{tcolorbox}
\usepackage{fancyhdr}
\usepackage{graphicx}
\usepackage{float}
\usepackage{bbm}

\setlength{\parindent}{0em}
\setlength{\parskip}{0.4em}

\definecolor{theoremblue}{RGB}{1, 73, 124}
\definecolor{corollaryblue}{RGB}{70, 143, 175}
\definecolor{exampleblue}{RGB}{137, 194, 217}

\newtcolorbox{tbox}{colback=theoremblue!20,colframe=theoremblue,
boxrule=0pt,arc=0pt,boxsep=2pt,left=2pt,right=2pt,leftrule=2pt}

\newtcolorbox{cbox}{colback=corollaryblue!20,colframe=corollaryblue,
boxrule=0pt,arc=0pt,boxsep=2pt,left=2pt,right=2pt,leftrule=2pt}

\newtcolorbox{ebox}{colback=exampleblue!20,colframe=exampleblue,
boxrule=0pt,arc=0pt,boxsep=2pt,left=2pt,right=2pt,leftrule=2pt}

\title{WuS - Lecture Notes Week 11}
\author{Ruben Schenk, ruben.schenk@inf.ethz.ch}
\date{\today}

\pagestyle{fancy}
\fancyhf{}
\rhead{ruben.schenk@inf.ethz.ch}
\rfoot{Page \thepage}
\lhead{WuS - Lecture Notes Week 11}

\begin{document}

\maketitle

\section{Konfidenzintervalle}

Die Grundidee ist wie folgt: Wie im vorigen Abschnitt suchen wir aus einer Familie $(\mathbb{P}_{\theta})_{\theta \in \Theta}$ von Modellen eines, das zu unseren Daten $x_1,..., \, x_n$ passt. Ein Schätzer für $\theta$ gibt uns dabei einen einzelnen zufälligen möglichen Parameterwert. Weil es schwirig ist, mit diesem einen Wert den richtigen Parameter zu treffen, suchen wir nun stattdessen eine \textbf{zufällige Teilmenge des Parameterbereichs,} die hoffentlich den wahren Parameter enthält.

\subsection{Definitionen}

Eir reichhaltig sind diese Schätzer? Werfen wir zum Beispiel eine Münze 100 mal, ohne die Wahrscheinlichkeit $p$ von Kopf zu kennen. Falls wir 70 mal Kopf erhalten, ist der Maximum-Likelihood-Schätzer für $p$ $T_{ML} = 0.7$. Wie weit ligt $T_{ML}$ von dem wahren Wert $p$ entfernt? Um diese Art von Fragen zu beantworten, führen wir den Begriff der Konfidenzintervalle ein.

\textbf{Def:} Sei $\alpha \in [0, \, 1]$. Ein \textbf{Konfidenzintervall für} $\theta$ \textbf{mit Niveau} $1 - \alpha$ ist ein Zufallsintervall $I = [A, \, B]$, sodass gilt
\[
    \forall \theta \in \Theta \quad \mathbb{P}_{\theta}[A \leq \theta \leq B] \geq 1 - \alpha,
\]
wobei $A, \, B$ Zufallsvariablen der Form $A = a(X_1,..., \, X_n)$, $B = b(X_1,..., \, X_n)$ mittels $a,b : \mathbb{R}^n \to \mathbb{R}$ sind.

\begin{ebox}
    \textbf{Beispiel (Konfidenzintervall für normales Modell mit bekannter Varianz):} Seien $X_1,..., \, X_n$ u.i.v. normalverteilte Zufallsvariablen mit Parametern $m$ und $\sigma^2 = 1.$ Wir betrachten somit ein stochastisches Modell mit bekannter Varianz ($\sigma^2 = 1$) aber unbekanntem Mittelwert $\mu$ ($X_1 \sim \mathcal{N}(\mu, \, 1)$). Mann kann zeigen, dass der Maximum-Likelihood Schätzer gegeben ist durch
    \[
        T = T_{ML} = \frac{X_1 + \cdots + X_n}{n},
    \]
    mit
\end{ebox}

\end{document}