\documentclass[a4paper]{extarticle}
\usepackage[utf8]{inputenc}
\usepackage[a4paper, margin=1in]{geometry}

\usepackage{amssymb}
\usepackage{amsmath}
\usepackage{enumitem}
\usepackage{tcolorbox}
\usepackage{fancyhdr}
\usepackage{graphicx}
\usepackage{float}

\graphicspath{ {./../images/} }

\setlength{\parindent}{0em}
\setlength{\parskip}{0.4em}

\definecolor{theoremblue}{RGB}{1, 73, 124}
\definecolor{corollaryblue}{RGB}{70, 143, 175}
\definecolor{exampleblue}{RGB}{137, 194, 217}

\newtcolorbox{tbox}{colback=theoremblue!20,colframe=theoremblue,
boxrule=0pt,arc=0pt,boxsep=2pt,left=2pt,right=2pt,leftrule=2pt}

\newtcolorbox{cbox}{colback=corollaryblue!20,colframe=corollaryblue,
boxrule=0pt,arc=0pt,boxsep=2pt,left=2pt,right=2pt,leftrule=2pt}

\newtcolorbox{ebox}{colback=exampleblue!20,colframe=exampleblue,
boxrule=0pt,arc=0pt,boxsep=2pt,left=2pt,right=2pt,leftrule=2pt}

\title{WuS - Lecture Notes Week 2}
\author{Ruben Schenk, ruben.schenk@inf.ethz.ch}
\date{\today}

\pagestyle{fancy}
\fancyhf{}
\rhead{ruben.schenk@inf.ethz.ch}
\rfoot{Page \thepage}
\lhead{WuS - Lecture Notes Week 2}

\begin{document}

\maketitle
\newpage

\subsection{Examples of Probability Space}

\subsubsection{Example with \(\Omega\) Finite}

We discuss a particular type of probability spaces where the sample space \(\Omega\) is an arbitrary \textbf{finite} set, and all the outcomes have the \textbf{same} probability \(p_{\omega} = \frac{1}{|\Omega|}\).

\textbf{Definition:} Let \(\Omega\) be a finite sample space. The \textbf{Laplace model} on \(\Omega\) is the triple \((\Omega, \, \mathcal{F}, \, \mathbb{P})\), where:

\begin{itemize}
    \item \(\mathcal{F} = \mathcal{P}(\Omega)\),
    \item \(\mathbb{P} : \mathcal{F} \to [0, \, 1]\) is defined by
\end{itemize}
\[
    \forall A \in \mathcal{F} \quad \mathbb{P}[A] = \frac{|A|}{|\Omega|}
\]

\begin{ebox}
    \textbf{Example:} We consider \(n \geq 3\) points on a circle, from which we select \(2\) at random. What is the probability that these two points selected are neighbors? We consider the Laplace model one
    \[
        \Omega = \{E \subset \{1, \, 2,..., \, n\} : |E| = 2\}.
    \]
    The event "the two points of \(E\) are neighbors" is given by
    \[
        A = \{\{1, \, 2\}, \, \{2, \, 3\},..., \, \{n-1, \, n\}, \, \{n, \, 1\}\}
    \]
    and we have
    \[
        \mathbb{P}[A] = \frac{|A|}{|\Omega|} = \frac{n}{\binom{n}{2}} = \frac{2}{n - 1}.
    \]
\end{ebox}

\subsubsection{Example with \(\Omega\) Infinite Countable}

\textbf{Example:} We throw a biased coin multiple times, at each throw, the coin falls on head with probability \(p\), and it falls on tail with probability \(1 - p\) (\(p\) is a fixed parameter in \([0, \, 1]\)). We stop at the first time we see a tail. The probability that we stop exactly at time \(k\) is given by
\[
    p_k = p^{k-1}(1-p).
\]

For this experiment, one possible probability space is given by:

\begin{itemize}
    \item \(\Omega = \mathbb{N} \setminus \{0\} = \{1, \, 2, \, 3,...\}\)
    \item \(\mathcal{F} = \mathcal{P}(\Omega)\)
    \item for \(A \in \mathcal{F}, \, \mathbb{P}[A] = \sum_{k \in A}p_k\)
\end{itemize}

\subsection{Properties of Events}

\subsubsection{Operations on Events and Interpretation}

The following propositions asserts that the different well-known set operations are allowed.

\begin{tbox}
    \textbf{Proposition (Consequences of the definition):} Let \(\mathcal{F}\) be a sigma-algebra on \(\Omega\). We have:

    \begin{itemize}
        \item \textbf{P4.} \(\emptyset \in \mathcal{F}\)
        \item \textbf{P5.} \(A_1, \, A_2,... \in \mathcal{F} \implies \bigcap_{i = 1}^{\infty} A_i \in \mathcal{F}\)
        \item \textbf{P6.} \(A, \, B \in \mathcal{F} \implies A \cup B \in \mathcal{F}\)
        \item \textbf{P7.} \(A, \, B \in \mathcal{F} \implies A \cap B \in \mathcal{F}\)
    \end{itemize}
\end{tbox}

A short summary of the common set-operations is given below:

\begin{itemize}
    \item \(A^C :\) \(A\) does not occur.
    \item \(A \cap B :\) \(A\) and \(B\) occur.
    \item \(A \cup B :\) \(A\) or \(B\) occurs
    \item \(A \Delta B :\) one and only one of \(A\) or \(B\) occurs
    \item \(A \subset B :\) If \(A\) occurs, then \(B\) occurs
    \item \(A \cap B = \emptyset :\) \(A\) and \(B\) cannot occur at the same time
    \item \(\Omega = A_1 \cup A_2 \cup A_3\) with \(A_1, \, A_2, \, A_3\) pairwise disjoint: for each outcome \(\omega\), one and only one of the events \(A_1, \, A_2, \, A_3\) is satisfied.
\end{itemize}

\subsection{Properties of Probability Measures}

\subsubsection{Direct Consequences of the Definition}

\begin{tbox}
    \textbf{Proposition:} Let \(\mathbb{P}\) be an arbitrary measure on \((\Omega, \, \mathcal{F})\). We have:

    \begin{itemize}
        \item \textbf{P3.} \(\mathbb{P}[\emptyset] = 0\).
        \item \textbf{P4. (additivity)} Let \(k \geq 1\). let \(A_1,..., \, A_k\) be \(k\) pairwise disjoint events, then \(\mathbb{P}[A_1 \cup \cdots \cup A_k] = \mathbb{P}[A_1] + \cdots + \mathbb{P}[A_k]\).
        \item \textbf{P5.} Let \(A\) be an event, then \(\mathbb{P}[A^C] = 1 - \mathbb{P}[A]\).
        \item \textbf{P6.} If \(A\) and \(B\) are two events (not necessarily disjoin), then \(\mathbb{P}[A \cup B] = \mathbb{P}[A] + \mathbb{P}[B] - \mathbb{P}[A \cap B]\).
    \end{itemize}
\end{tbox}

\subsubsection{Useful Inequalities}

\textbf{Proposition (Monotonicity):} Let \(A, \, B \in \mathcal{F}\), then
\[
    A \subset B \implies \mathbb{P}[A] \leq \mathbb{P}[B].
\]

\textbf{Proposition (Union bound):} Let \(A_1, \, A_2,...\) be a sequence of events (not necessarily disjoint), then we have
\[
    \mathbb{P}[\bigcup_{i = 1}^{\infty} A_i] \leq \sum_{i = 1}^{\infty} \mathbb{P}[A_i].
\]

\textbf{Remark:} The union bound also applies to a \textit{finite} collection of events.

\subsubsection{Continuity Properties of Probability Measures}

\begin{tbox}
    \textbf{Proposition:} Let \((A_n)\) be an increasing sequence of events (i.e. \(A_n \subset A_{n + 1}\) for every \(n\)). then
    \[
        \lim_{n \to \infty} P[A_n] = \mathbb{P}[\bigcup_{n = 1}^{\infty}A_n]. \quad \text{(increasing limit)}
    \]
    Let \((B_n)\) be a decreasing sequence of events (i.e. \(B_n \supset B_{n + 1}\) for every \(n\)). Then
    \[
        \lim_{n \to \infty} P[B_n] = \mathbb{P}[\bigcap_{n = 1}^{\infty} B_n]. \quad \text{(decreasing limit)}
    \]
\end{tbox}

\textbf{Remark:} By monotonicity, we have \(\mathbb{P}[A_n] \leq \mathbb{P}[A_{n + 1}]\) and \(\mathbb{P}[B_n] \geq \mathbb{P}[B_{n + 1}]\) for every \(n\). Hence the limits in the proposition are well defined as monotone limits.

\subsection{Conditional Probabilities}

\textbf{Definition (Conditional probability):} Let \((\Omega, \, \mathcal{F}, \, \mathbb{P})\) be some probability space. Let \(A, \, B\) be two events with \(\mathbb{P}[B] > 0\). The \textbf{conditional probability of} \(A\) \textbf{given} \(B\) is defined by
\[
    \mathbb{P}[A \, | \, B] = \frac{\mathbb{P}[A \cap B]}{\mathbb{P}[B]}.
\]

\textbf{Remark:} \(\mathbb{P}[B \, | \, B] = 1\).

\textbf{Proposition:} Let \(\Omega, \, \mathcal{F}, \, \mathbb{P}\) be some probability space. Let \(B\) be an event with positive probability. Then \(\mathbb{P}[ \,. \, | \, B]\) is a probability measure on \(\Omega\).

\begin{tbox}
    \textbf{Proposition (Formula of total probability):} Let \(B_1,..., \, B_n\) be a partition of the sample space \(\Omega\) with \(\mathbb{P}[B_i] > 0\) for every \(1 \leq i \leq n\). Then, one has
    \[
        \forall A \in \mathcal{F} : \mathbb{P}[A] = \sum_{i = 1}^n \mathbb{P}[A \, | \, B_i] \mathbb{P}[B_i].
    \]
    Here, a \textit{partition} \(B_i\) is such that \(\Omega = B_1 \cup \cdots \cup B_n\) and the events are pariwise disjoint.
\end{tbox}

\begin{tbox}
    \textbf{Proposition (Bayes formula):} Let \(B_1,..., \, B_n \in \mathcal{F}\) be a partition of \(\Omega\) with \(\mathbb{P}[B_i] > 0\) for every \(i\). For every event \(A\) with \(\mathbb{P}[A] > 0\), we have
    \[
        \forall i = 1,..., \, n : \mathbb{P}[B_i \, | \, A] = \frac{\mathbb{P}[A \, | \, B_i] \cdot \mathbb{P}[B_i]}{\sum_{j = 1}^n \mathbb{P}[A \, | \, B_j] \cdot \mathbb{P}[B_j]}.
    \]
\end{tbox}

\subsection{Independence}

\subsubsection{Independence of Events}

\textbf{Definition (Independence of two events):} Let \((\Omega, \, \mathcal{F}, \, \mathbb{P})\) be a probability space. Two events \(A\) and \(B\) are said to be \textbf{independent} If
\[
    \mathbb{P}[A \cap B] = \mathbb{P}[A] \cdot \mathbb{P}[B].
\]

\textbf{Remark:} If \(\mathbb{P}[A] \in \{0, \, 1\}\), then \(A\) is independent of every event, i.e. \(\forall B \in \mathcal{F} : \mathbb{P}[A \cap B] = \mathbb{P}[A] \cdot \mathbb{P}[B]\). Furthermore we might also state, that \(A\) is independent of \(B\) if and only if \(A\) is independent of \(B^C\).

\textbf{Proposition:} Let \(A, \, B \in \mathcal{F}\) be two events with \(\mathbb{P}[A], \, \mathbb{P}[B] > 0\). Then the following are equivalent:

\begin{itemize}
    \item \(\mathbb{P}[A \cap B] = \mathbb{P}[A] \cdot \mathbb{P}[B]\) (\(A\) and \(B\) are independent)
    \item \(\mathbb{P}[A \, | \, B] = \mathbb{P}[A]\) (the occurrence of \(B\) has no influence on \(A\))
    \item \(\mathbb{P}[B \, | \, A] = \mathbb{P}[B]\) (the occurrence of \(A\) has no influence on \(B\))
\end{itemize}

\textbf{Definition:} Let \(I\) be an arbitrary set of indices. A collection of events \((A_i)_{i \in I}\) is said to be \textbf{independent} if
\[
    \forall J \subset I \text{ infinite} : \mathbb{P}[\bigcap_{j \in J} A_j] = \prod_{j \in J} \mathbb{P}[A_j].
\]

\end{document}