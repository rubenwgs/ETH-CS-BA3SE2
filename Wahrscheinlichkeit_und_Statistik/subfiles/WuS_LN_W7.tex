\documentclass[a4paper]{extarticle}
\usepackage[utf8]{inputenc}
\usepackage[a4paper, margin=1in]{geometry}

\usepackage{amssymb}
\usepackage{amsmath}
\usepackage{enumitem}
\usepackage{tcolorbox}
\usepackage{fancyhdr}
\usepackage{graphicx}
\usepackage{float}
\usepackage{bbm}

\setlength{\parindent}{0em}
\setlength{\parskip}{0.4em}

\definecolor{theoremblue}{RGB}{1, 73, 124}
\definecolor{corollaryblue}{RGB}{70, 143, 175}
\definecolor{exampleblue}{RGB}{137, 194, 217}

\newtcolorbox{tbox}{colback=theoremblue!20,colframe=theoremblue,
boxrule=0pt,arc=0pt,boxsep=2pt,left=2pt,right=2pt,leftrule=2pt}

\newtcolorbox{cbox}{colback=corollaryblue!20,colframe=corollaryblue,
boxrule=0pt,arc=0pt,boxsep=2pt,left=2pt,right=2pt,leftrule=2pt}

\newtcolorbox{ebox}{colback=exampleblue!20,colframe=exampleblue,
boxrule=0pt,arc=0pt,boxsep=2pt,left=2pt,right=2pt,leftrule=2pt}

\title{WuS - Lecture Notes Week 7}
\author{Ruben Schenk, ruben.schenk@inf.ethz.ch}
\date{\today}

\pagestyle{fancy}
\fancyhf{}
\rhead{ruben.schenk@inf.ethz.ch}
\rfoot{Page \thepage}
\lhead{WuS - Lecture Notes Week 7}

\begin{document}

\maketitle

\subsubsection{Randverteilung}

Unter Kenntnis der Verteilung von $X_1,..., \, X_n$ kann man die Verteilung der einzelnen $X_i$ separat ermitteln. In diesem Zusammenhang wird die Verteilung von $X_i$ als $i$-te \textbf{Randverteilung} bezeichnet.

\begin{cbox}
    \textbf{Satz:} Seien $X_1,..., \, X_n$ diskrete Z.V. mit gemeinsamer Verteilung $p = (p(x_1,..., \, x_n))_{x_i \in W_1,..., \, x_m \in W_n}$. Für jedes $i$ gilt:
    \[
        \forall z \in W_i \quad \mathbb{P}[X_i = z] = \sum_{x_1,..., \, x_{i-1}, \, x_{i + 1},..., \, x_n} p(x_1,..., \, x_{i-1}, \, z, \, x_{i+1},..., \, x_n).
    \]
\end{cbox}

\subsubsection{Unabhängigkeit}

\begin{cbox}
    \textbf{Satz:} Seien $X_1,..., \, X_n$ diskrete Zufallsvariablen mit gemeinsamer Verteilung $p = (p(x_1,..., \, x_n))_{x_1 \in W_1,..., \, x_n \in W_n}$. Die folgenden Aussagen sind äquivalent:

    \begin{enumerate}
        \item $X_1,..., \, X_n$ sind unabhängig.
        \item $p(x_1,..., \, x_n) = \mathbb{P}[X_1 = x_1] \cdots \mathbb{P}[X_n = x_n]$ für jedes $x_1 \in W_1,..., \, x_n \in W_n$.
    \end{enumerate}
\end{cbox}

\subsection{Stetige Gemeinsame Verteilung}

\subsubsection{Definition}

\textbf{Def:} Sei $n \geq 1$. Wir sagen, dass die Z.V. $X_1,...,X_n : \Omega \to \mathbb{R}$ eine \textbf{stetige gemeinsame Verteilung} besitzen, falls eine Abbildung $f : \mathbb{R}^n \to \mathbb{R}_+$ existiert, sodass
\[
    \mathbb{P}[X_1 \leq a_1,..., \, X_n \leq a_n] = \int_{- \infty}^{a_1} \cdots \int_{-\infty}^{a_n} f(x_1,..., \, x_n) \, dx_n ... dx_1
\]
für jedes $a_1,..., \, a_n \in \mathbb{R}$ gilt. Obige Abbildung $f$ nennen wir gerade \textbf{gemeinsame Dichte von} $(X_1,..., \, X_n)$.

\begin{cbox}
    \textbf{Satz:} Sei $f$ die gemeinsame Dichte der Zufallsvariablen $(X_1,..., \, X_n)$. Dann gilt
    \[
        \int_{- \infty}^{\infty} \cdots \int_{ - \infty}^{\infty} f(x_1,...,x_n) \, dx_n ... dx_1 = 1.
    \]
\end{cbox}

\begin{ebox}
    \textbf{Intuition:} Nehmen wir zum Beispiel zwei Z.V. $X, \, Y$. Intuitiv beschreibt $f(x,y) \, dxdy$ die Wahrscheinlichkeit, dass ein Zufallspunkt $(X, \, Y)$ in einem Rechteck $[x, \, x + dx] \times [y, \, y + dy]$ liegt.
\end{ebox}

\subsubsection{Erwartungswert unter Abbildungen}

\begin{cbox}
    \textbf{Satz:} Sei $\phi : \mathbb{R}^n \to \mathbb{R}$ eine Abbildung. Falls $x_1,..., \, X_n$ eine gemeinsame Dichte $f$ besitzen, dann lässt sich der Erwartungswert der Z.V. $Z = \phi(X_1,..., \, X_n)$ mittels
    \[
        \mathbb{E}[Z] = \int_{- \infty}^{\infty} \cdots \int_{- \infty}^{\infty} \phi(x_1,..., \, x_n) \cdot f(x_1,..., \, x_n) \, dx_1...dx_n,
    \]
    berechnen (solange das Integral wohldefiniert ist).
\end{cbox}

\begin{ebox}
    \textbf{Beispiel:} Betrachten wir das Paar $(X, \, Y)$ analog zum obigen Beispiel. Falls wir die Funktion $\phi(x, \, y) = \mathbbm{1}_{(x, \, y) \in R}$ betrachten, gilt für jedes Rechteck $R = (a, \, a') \times (b, \, b') \subseteq [0, \, 1]^2$:
    \[
        \mathbb{P}[(X, \, Y) \in R] = \mathbb{E}[\phi(X, \, Y)] = \int_a^{a'} \int_b^{b'} dxdy = (a'-a)(b'-b) = \text{Flaeche}(R).
    \]
\end{ebox}

\subsubsection{Randverteilungen}

Falls $X. \, Y$ eine gemeinsame Dichte $f_{X, \, Y}$ besitzt, dann gilt
\begin{align*}
    \mathbb{P}[X \leq a] &= \mathbb{P}[X \in [- \infty, \, a], \, Y \in [- \infty, \, \infty]]\\ &= \int_{- \infty}^a \Big(\int_{- \infty}^{\infty} f(x, \, y)\,dy \Big)dx.
\end{align*}
Somit is $X$ stetig mit folgender Dichte:
\[
    f_X(x) = \int_{- \infty}^{\infty} f(x, \, y) \, dy.
\]
Analog ist $Y$ stetig mit folgender Dichte:
\[
    f_Y(y) = \int_{- \infty}^{\infty} f(x, \, y) \, dx.
\]

\textbf{Bemerkung:} Folgende Implikationen gelten:
\[
    X, \, Y \text{ diskrete Z.V.} \iff X, \, Y \text{ gemeinsame diskrete Z.V.}
\]
\[
    X, \, Y \text{ gemeinsam stetig} \implies X \text{ stetig und } Y \text{ stetig}.
\]

\begin{ebox}
    \textbf{Beispiel:} Schauen wir uns die Gleichverteilung eines Punktes auf einem Quadrat an. Unter gemeinsamer Dichte $f_{X,Y}(x, \, y) = \mathbbm{1}_{0 \leq x,y \leq 1}$ hat $X$ folgende Dichte:
    \[
        f_X(x) = \int_0^1 \mathbbm{1}_{0 \leq x \leq 1} \mathbbm{1}_{0 \leq y \leq 1} \, dy = \mathbbm{1}_{0 \leq x \leq 1}.
    \]
    Analog ist $f_Y(y) = \mathbbm{1}_{0 \leq y \leq 1}$. Somit sind sowohl $X$ als auch $Y$ gleichverteilte Zufallsvariablen auf $[0, \, 1]$ ($\mathcal{U} \sim [0, \, 1]$).
\end{ebox}

\subsubsection{Unabhängigkeit stetiger Zufallsvariablen}

\begin{tbox}
    \textbf{Theorem:} Seien $X_1,..., \, X_n$ Z.V. mit Dichten $f_1,..., \, f_n$. Dann sind folgende Aussagen äquivalent:
    \begin{enumerate}
        \item $X_1,..., \, X_n$ sind unabhängig,
        \item $X_1,..., \, X_n$ sind insgesamt stetig mit gemeinsamer Dichte.
    \end{enumerate}
    \[
        f(x_1,..., \, x_n) = f_1(x_1) \cdots f_n(x_n)
    \]
\end{tbox}

\textbf{Bemerkung:} Somit sind zwei unabhängige stetige Z.V. automatisch gemeinsam stetig.

\end{document}