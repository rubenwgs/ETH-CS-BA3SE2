\documentclass[a4paper]{extarticle}
\usepackage[utf8]{inputenc}
\usepackage[a4paper, margin=1in]{geometry}

\usepackage{amssymb}
\usepackage{amsmath}
\usepackage{enumitem}
\usepackage{tcolorbox}
\usepackage{fancyhdr}
\usepackage{graphicx}
\usepackage{float}

\setlength{\parindent}{0em}
\setlength{\parskip}{0.4em}

\definecolor{theoremblue}{RGB}{1, 73, 124}
\definecolor{corollaryblue}{RGB}{70, 143, 175}
\definecolor{exampleblue}{RGB}{137, 194, 217}

\newtcolorbox{tbox}{colback=theoremblue!20,colframe=theoremblue,
boxrule=0pt,arc=0pt,boxsep=2pt,left=2pt,right=2pt,leftrule=2pt}

\newtcolorbox{cbox}{colback=corollaryblue!20,colframe=corollaryblue,
boxrule=0pt,arc=0pt,boxsep=2pt,left=2pt,right=2pt,leftrule=2pt}

\newtcolorbox{ebox}{colback=exampleblue!20,colframe=exampleblue,
boxrule=0pt,arc=0pt,boxsep=2pt,left=2pt,right=2pt,leftrule=2pt}

\title{EnpRisk - Lecture Notes Week 6}
\author{Ruben Schenk, ruben.schenk@inf.ethz.ch}
\date{\today}

\pagestyle{fancy}
\fancyhf{}
\rhead{ruben.schenk@inf.ethz.ch}
\rfoot{Page \thepage}
\lhead{EnpRisk - Lecture Notes Week 6}

\begin{document}

\maketitle

\subsubsection{Generalized Logistic Growth}

Many systems exhibit succession of S-curves because the advances in technology etc. increase the carrying capacity $K$. One idea to \textit{generalize the logistic growth} is to include into this the logistic equation with a population dependent carrying capacity with \textit{delay time} $\tau$:

\[
    \frac{dP}{dt} = rP(t) \big[1 - \frac{P(t)}{K(t)}\big], \text{ with } K(t) = A + BP(t - \tau)
\]

In other words:

\[
    \frac{dx}{dt} = \alpha \cdot x(t) + \beta \cdot \frac{x^2(t)}{a + bx(t - \tau)},
\]

with $x \sim P$ and parameters $a, \, b$ related to $r, \, A$ and $B$. We can distinguish four different scenarios:

\begin{itemize}
    \item $\alpha = +1, \, \beta = -1:$ gain and competition
    \item $\alpha = +1, \, \beta = +1:$ gain and cooperation
    \item $\alpha = -1, \, \beta = -1:$ loss and competition
    \item $\alpha = -1, \, \beta = +1:$ loss and cooperation
\end{itemize}

Another idea is to instead of a linearly growing capacity, consider the case of exponential growth:

\[
    \frac{dx}{dt} = \sigma_1 x(t) - \sigma_2 \frac{x^2}{y(x)}, \text{ with } y(x) = \exp(bx(t - \tau))
\]

In other words:

\[
    \frac{dx}{dt} = \sigma_1 x(t) - \sigma_2 x^2(t)e^{-bx(t - \tau)}
\]

We can distinguish the same cases as before with $\sigma_1 = \pm 1$ and $\sigma_2 = \pm 1$.

Finally, we can also consider \textbf{coupled logistic equations.} The idea is, that instead of only one species, we can also have two interacting species:

\[
    \frac{dx}{dt} = x - \frac{x^2}{1 + bxz} \quad \frac{dz}{dt} = z - \frac{z^2}{1 + gxz},
\]

where $x$ is the species one and $y$ is the species two.

\subsubsection{Interlude: Chaos Theory}

The \textbf{logistic map} is defined as:

\[
    x(n + 1) = \alpha x(n) [1 - x(n)]
\]

It can be shown that $x$ is \textit{chaotic} for almost all values of $\alpha \in [3.569..., \, 4]$.

It can also be shown that the logistic map is nothing but a discrete version of the logistic equation

\[
    \frac{P(t)}{dt} = rP(t) \big[1 - \frac{P(t)}{K} \big],
\]

with $\alpha = r + 1$.

To distinguish between stochastic (completely random) and chaotic maps, we can plot the values $x(n + 1)$ as a function of $x(n)$. The phase-space of a chaotic map is low dimensional!

\end{document}